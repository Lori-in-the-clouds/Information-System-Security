\chapter{Security Implementations in OSI Levels}
\begin{quotebox-yellow}{}
    In the ISO/OSI stack, there is \textbf{not} a single optimal level in which we can implement security. Typically, L6 (Presentation) is the \textbf{only} level in which security measures are \textbf{not} useful.
\begin{itemize}
    \item The \textbf{higher} we go in the stack, the more our security features are \textbf{effective}, but we \textbf{slow}
    down our system and we leave more room for DoS attacks.
    \item The \textbf{lower} we go in the stack, the more \textbf{quickly} we detect attacks, but the \textbf{fewer} are he data available to us to detect them.
\end{itemize}
\end{quotebox-yellow}
\section{DHCP Security}
When L3 is reached, one of the first protocols that is activated is the \textbf{DHCP (Dynamic Host Configuration Protocol)}. Unfortunately, DHCP is \textbf{not} capable of performing peer authN
and uses \textbf{broadcast} packets that carry: IP address; netmask; default gateway; local DNS and local DNS suffix (\(\rightarrow \) \textbf{sensitive data}).\\   
\\    
Since DHCP Requests are L2 broadcast frames, an attacker could activate a \textbf{Fake DHCP
Server} by staying in the same broadcast domain of the victim and Sniffing the victim’s DHCP Request.\\    
\\    
Possible \textcolor{red}{\underline{\textbf{attacks}}} that can be performed by a Fake DHCP Server are:
\begin{itemize}
    \item \textbf{DoS}: done by providing a wrong network configuration to the victim, denying access to the network.
    \item \textbf{MIMT}:  a valid IP address is provided to the victim, but it will be a part of a /30 subnet (\(\rightarrow \) \textbf{only} two IP addresses can be assigned):
    \begin{itemize}
        \item One is given to the user
        \item The other is configured by the attacker as the victim’s default gateway
    \end{itemize}
    In this way, the victim is \textbf{ogically isolated} in a subnet of its own \(\rightarrow \) to communicate the victim has to send \textbf{everything} through the attacker.\\   
    \\
However, with this configuration, \textbf{incoming traffic} could still get to the victim \textbf{without} passing through the attacker. The attacker however could also activate NAT to successfully intercept the incoming traffic.
\item \textbf{Malicious Name-Address Translation}: the attacker declares himself as \textbf{Local DNS}. Then, when the victim wants to perform a \textbf{name} \(\rightarrow \) \textbf{address} translation, the attacker
will provide the wrong address. This tactic is usually used for Phishing and Pharming attacks;
\end{itemize}
CISCO tried to \textcolor{green}{\textbf{solve}} these security problems by implementing into switches: 
\begin{itemize}
    \item \textbf{DHCP Snooping}: the switch acepts \textbf{only} DHCP Reply from \textbf{trusted} ports.
    \item \textbf{IP Guard}: switching is performed \textbf{only} for IP addresses that have been issued from a
    \textbf{valid} DHCP Server.
\end{itemize}
\textcolor{red}{\textbf{N.B.}} There is a memory limit on how many valid IPs can be stored in the switch.
\subsection{AuthN for DHCP Messages}
\textbf{RFC-3118} uses HMAC-MD5 to perform \textbf{data authN} the DHCP Frames, but, given its complexity, it is rarely adopted: since HMAC is a Symmetric Protocol, a secret key needs to be installed \textbf{manually} on each machine that has to use DHCP and on each DHCP server. Moreover, since the key is Symmetric there is \textbf{no} way to distinguish between a DHCP Client and a DHCP Server.
\newpage
\section{VPN (Virtual Private Network)}
\begin{minipage}{0.6\textwidth}
%	\vspace{-0.5cm}
L3 is the first layer to provide \textbf{End-to-End Connectivity}, so it is possible to create both
\textbf{End-to-End Protection} and \textbf{VPN}s. End-to-End Protection is needed for \textbf{securing data} as soon as they exit from the node that has generated them, until they reach the final network interface. This means that the \textbf{only} possible attacks are those inside the node, allowing us to forget about all the other attacks that come from L3 (apart from DoS). 
\end{minipage} 
\hspace{0.3cm}
\begin{minipage}{0.4\textwidth}
    \centering
    \includegraphics[width=0.9\textwidth]{/home/lorenzo/Pictures/Screenshots/Screenshot from 2024-12-30 16-40-45.png}
\end{minipage}
\noindent
VPN is a technique (implemented via hardware and/or software) used to create a \textbf{Private Network} while using shared (possibly \textbf{untrusted}) channels and/or communication devices.
\begin{figure}[H]
    \centering
    \includegraphics[width=0.5\textwidth]{/home/lorenzo/Notes/Information System Security/images/Screenshot from 2024-12-30 16-45-35.png}
    \caption{VPNs in a Public Network}
\end{figure}
A VPN can be created in three different ways:
\begin{itemize}
    \item \underline{\textbf{VPN via Private Address}}: The networks that want to be part of the VPN \textbf{must} use \textbf{Non-Public IP Addresses} that are \textbf{unreachable} from the other networks. These VPNs are considered \textbf{private} since they do not require authorization and the packets are \textbf{not} globally routable.\\   
    However, these protections can be easily defeated if somebody:
    \begin{itemize}
        \item Guesses or discovers the non-public IP addresses.
        \item Sniffs the packets during their transmission.
        \item Manages to access the communication devices.
    \end{itemize}
    \vspace{-0.2cm}
    \begin{quotebox-red}{Beware}
        This kind of VPN does not
        implement any type of protection for packets, users or the infrastructure itself \(\rightarrow \) the level of security of this solution is close to \textbf{zero}.
    \end{quotebox-red}
    \item \underline{\textbf{VPN via Tunnel}}: The routers encapsulate the whole L3 packet as the payload of another L3 packet (e.g. \textbf{IP in
    IP, IP over MPLS}, etc...). \textbf{Before} the encapsulation, the Border Routers perform \textbf{access} control to the VPN by checking an \textbf{ACL} (\textbf{Access Control List}).  
    For example, if the VPN belongs to the 10.1.0.0\textbackslash16 address range, the VPN can be accessed \textbf{only} by someone in 10.1.0.0\textbackslash16.\\    
    \\
    This solution grants VPN Providers protection against \textbf{malicious} VPN Customers, because it is impossible for Customers can change the subnet to which they belong (if they try, they do not belong anymore to the VPN).

    \begin{quotebox-red}{Beware}
        However, this protection can be defeated by anybody that \textbf{manages} a router or by Sniffing attacks \(\rightarrow \) protection works for Providers but \textbf{not} for Customers.
    \end{quotebox-red}
    \begin{quotebox-grey}{Example: VPN via IP Tunnel}
    \begin{minipage}{0.6\textwidth}
    %	\vspace{-0.5cm}
    \textbf{Net 1} and \textbf{Net 2} have got the \textbf{same color} because they belong to the \textbf{same subnet}.
    This VPN uses IP in IP Tunneling and works in the following way: 
    \begin{enumerate}
        \item \textbf{Node A} from \textbf{Net 1} sends a packet to \textbf{Node B} in \textbf{Net 2}.
        \item . The packet reaches the Border Router \textbf{R1}, which encapsulates it by adding the \textbf{IPv4 Header (Tunnel)}.
        \item The Tunneled Packet is forwarder from \textbf{R1} to \textbf{R2}.
        \item \textbf{R2} decapsulates the packet and forwards it to \textbf{Node 2}.
    \end{enumerate}
    \end{minipage} 
    \hspace{0.2cm}
    \begin{minipage}{0.4\textwidth}
        \centering
        \includegraphics[width=0.9\textwidth]{/home/lorenzo/Notes/Information System Security/images/Screenshot from 2024-12-30 17-38-10.png}
    \end{minipage}
    \end{quotebox-grey}
    \item \underline{\textbf{VPN via Secure IP Tunnel}}: This type of VPN, also known as \textbf{S-VPN} (\textbf{Secure VP}), protects \textbf{also} VPN Customers. To
    do so, before encapsulation, the packets are protected with:
    \begin{itemize}
        \item \textbf{MAC}: grants \textbf{integrity} and \textbf{data authN}.
        \item \textbf{Encryption}: confidentiality.
        \item \textbf{Numbering}: to avoid Replay attacks.
    \end{itemize}
    There is \textbf{no} Digital Signature, because Asymmetric Encryption is \textbf{slow} and does \textbf{not} fit the
speed standards of modern networks. If the selected algorithms are \textbf{strong} \(\rightarrow \) only DoS attacks are possible.\\
\begin{minipage}{0.6\textwidth}
    %	\vspace{-0.5cm}
        In this picture, there are the Border Routers \textbf{R1} and \textbf{R2}, and the \textbf{TAP}s (\textbf{Tunnel Access Point}) \textbf{T1} and \textbf{T2}:
        \begin{itemize}
            \item Routers performs \textbf{encapsulation} and \textbf{decapsulatation}
            \item TAPs perform \textbf{cryptographic} and \textbf{decapsulatation}.
        \end{itemize}
    \end{minipage} 
    \hspace{0.2cm}
    \begin{minipage}{0.4\textwidth}
        \centering
        \includegraphics[width=0.9\textwidth]{/home/lorenzo/Notes/Information System Security/images/Screenshot from 2024-12-30 17-59-07.png}
    \end{minipage}
    \begin{quotebox-red}{Beware}
        If the TAP is managed by an \textbf{external} network Provider \(\rightarrow \) this is \textbf{fake security}.\\
        Security can be achieved only if there are separated devices, where:
        \begin{itemize}
            \item The TAPs are managed by the \textbf{client}.
            \item The Border Routers are be managed by the \textbf{ISP}.
        \end{itemize}
    \end{quotebox-red}
\end{itemize}
\noindent{\color{gray!50}\rule{\textwidth}{0.5pt}}
\section{IPsec}
IPSec (Internet Protocol Security) is a suite of protocols used to secure IP communications. It used to create:
\begin{itemize}
    \item \textbf{S-VPN} over \textbf{unstrusted} networks.
    \item \textbf{End-to-End secure packet flows}
\end{itemize}
IPsec defines two specific packet types:
\begin{itemize}
    \item \textbf{AH} (\textbf{Authentication Header}): provides \textbf{integrity}, \textbf{data authN} and protection against Replay attacks.
    \item \textbf{ESP} (\textbf{Encapsulating Security Payload}): provides nearly the same functions as \textbf{AH} and additionally grants \textbf{payload confidentiality}.\\    \\
    \textcolor{red}{\textbf{N.B.}} \textbf{Confidentiality} can \textbf{never} be provided for the \textbf{header}, if it is encrypted \(\rightarrow \) the intermediate systems would \textbf{not} be able to process the packets.
\end{itemize}
\noindent
Moreover, IPsec also defines \textbf{IKE} (\textbf{Internet Key Exchange}), which is a dedicated protocol for Key-Exchange.
\\ Thanks to all of these features, IPsec offers the following Security Services:
\begin{itemize}
    \item \textbf{AuthN of IP Packets}: achieved with the computation of a Keyed-Digest of the packets
    with a symmetric key. This provides:
    \begin{itemize}
        \item \textbf{Data Integrity}: the receiver can detect if the packets have been manipulated.
        \item \textbf{Sender AuthN}:  formal proof of the sender’s identity.
        \item \textbf{Partial} protection against Replay attacks \(\rightarrow \) working at L3 means that packets can be lost or duplicated.
    \end{itemize}
    \item \textbf{Confidentiality of IP Packets} \(\rightarrow \) but only for the \textbf{payload}.
    \item \textbf{Peer AuthN when Creating the SA}: when creating a \textbf{Security Association} (\textbf{SA}), a Key Agreement is needed \textbf{after} peer authN.
\end{itemize}

\subsection{SA (Security Association)}
A \textbf{SA} is an \textbf{unidirectional} logic connection between two IPsec systems. Each SA has associated a \textbf{single} and \textbf{unique} Security Service. To achieve \textbf{full protection} for a bidirectional
packet flow between two nodes, we \textbf{always} need two SAs, one A → B and one B → A.
\begin{figure}[H]
    \centering
    \includegraphics[width=0.5\textwidth]{/home/lorenzo/Notes/Information System Security/images/Screenshot from 2024-12-30 18-53-51.png}
\end{figure}

\subsection{Local "Database"}
SAs are managed via two \textbf{Local "Databases"}:
\begin{itemize}
    \item \textbf{SPD (Security Policy Database)}: contains the list of \textbf{Security Policies} (\textbf{SP}) to apply to different packet flows.
    The SPD is configured \textbf{a-priori} (e.g. manually) or connected to an \textbf{automatic system} (e.g. ISPS (Internet Security Policy System)).
    \item \textbf{SAD (SA Database)}:  a runtime "database" that contains a list of the \textbf{active} SAs and their characteristics (e.g. algorithms, keys, parameters).
\end{itemize}

\subsection{Sending Packet with IPsec}
When a node wants to send a packet and IPsec is installed on that node, the following questions need to be answered:
\\
\\
\begin{minipage}{0.6\textwidth}
%	\vspace{-0.5cm}
    \begin{itemize}
        \item Which SP should be applied to this packet? \(\rightarrow \) check the SPD:
        \begin{itemize}
            \item If a SP is provided \(\rightarrow \) it is applied
            \item If \textbf{no} SP is provided \(\rightarrow \)  the packet goes straight to L2 for transmission.
        \end{itemize}
        \item Is this the first packet of the flow? → check the SAD:
        \begin{itemize}
            \item If an SA is present \(\rightarrow \) IPsec applies the associated parameters that are used to
            protect the packet and then sends it to L2 for transmission.
            \item If \textbf{no} corresponding SA is present \(\rightarrow \) IPsec creates the corresponding SA and proceeds as above.
        \end{itemize}
    \end{itemize} 
\end{minipage} 
\hspace{0.3cm}
\begin{minipage}{0.4\textwidth}
    \centering
    \includegraphics[width=0.9\textwidth]{/home/lorenzo/Notes/Information System Security/images/Screenshot from 2024-12-30 21-58-23.png}
\end{minipage}

\subsection{Transport Mode in IPsec}
\begin{minipage}{0.6\textwidth}
%	\vspace{-0.5cm}
Transport Mode is used for \textbf{End-to-End Security} by Hosts but \textbf{not} by Gateways.\\
The original packet is cut in two parts so that a new header (the \textbf{IPsec Header}) can be added
in between. The IPv4 header will say that it is transporting an IPsec Packet, while in the IPsec
Header there is a field that says what is the \textbf{actual} Transport Protocol (e.g. TCP, UDP) used
by the packet. 
\end{minipage} 
\hspace{0.3cm}
\begin{minipage}{0.4\textwidth}
    \centering
    \includegraphics[width=\textwidth]{/home/lorenzo/Notes/Information System Security/images/Screenshot from 2024-12-30 22-03-48.png}
\end{minipage}
\begin{itemize}
    \item \textcolor{green}{\textbf{Pro}}: it is \textbf{fast}.
    \item \textcolor{red}{\textbf{Con}}: \textbf{no} protection of the IPv4 header. 
\end{itemize}

\subsection{Tunnel Mode in IPsec}
\begin{minipage}{0.6\textwidth}
%	\vspace{-0.5cm}
\textbf{Tunnel Mode} is used to create a VPN, usually, among Gateways.\\
In Tunnel Mode, the entire original IP packet, including \textbf{both the header and payload}, is encapsulated within a new IP packet. This provides \textbf{comprehensive protection} for the original packet, as encryption and authentication are applied to the encapsulated content.
Since the entire original IP packet (including the header) is encapsulated, all variable fields of
the inner header are fully protected, ensuring confidentiality and integrity (\textbf{Protection of End-to-End (E2E) Header Fields}). 
\end{minipage} 
\hspace{0.5cm}
\begin{minipage}{0.4\textwidth}
    \centering
    \includegraphics[width=\textwidth]{/home/lorenzo/Notes/Information System Security/images/Screenshot from 2024-12-30 22-18-55.png}
\end{minipage}

\begin{center}
\vspace{-0.4cm}
\begin{quotebox-red}{Beware}
    The Secure IPsec Tunnel is \textbf{not} created among Routers, but between the Gateways,
    since Gateways are the actual \textbf{contact point} between a secure network and an unsecure one
    \(\rightarrow \) the Gateway has the role of adding protection by creating a Secure IPsec Tunnel
\end{quotebox-red}
\end{center}
\begin{itemize}
    \item \textcolor{green}{\textbf{Pro}}: full packet confidentiality.
    \item \textcolor{red}{\textbf{Con}}: it is \textbf{slow}.
\end{itemize}
\newpage
\subsection{AH (Authentication Header)}
There are two version of \textbf{AH}:
\begin{itemize}
    \item \textbf{1v Version}: it provides \textbf{integrity} and \textbf{sender authN}. System who use this version
    \textbf{must} support \textbf{Keyed-MD5} and can \textbf{optionally} support \textbf{Keyed-SHA1}.
    \item \textbf{2v Version}: it provides \textbf{integrity}, \textbf{sender authN} and \textbf{partial} protection from Replay
    attacks. System who use this version \textbf{must} support \textbf{HMAC-MD5-96} and \textbf{HMAC-SHA1-96}.
\end{itemize}
The format of the header that is being added to the IPsec packet has:
\\    
\\
\begin{minipage}{0.6\textwidth}
%	\vspace{-0.5cm}
    \begin{itemize}
        \item \textbf{Next Header} (1 byte): it contains the real transporting protocol of the packet, since in the IP header there will be written that it is transporting AH.
        \item \textbf{Lenght} (1 byte): describes the length of the packet.
        \item \textbf{Reserved} (2 bytes): for future use.
        \item \textbf{SPI} (\textbf{Security Parameter Index}, 4 bytes): refer in a quick and easy way to all the parameters that are needed to protect the packet.
        \item \textbf{Sequence Number} (4 bytes): it is different from the one in the IP header, as it is used to avoid Replay attacks.
        \item \textbf{ICV} (\textbf{Integrity Check Value}): Digest that provides \textbf{data authN}.
    \end{itemize}
\end{minipage} 
\hspace{0.3cm}
\begin{minipage}{0.4\textwidth}
    \centering
    \includegraphics[width=1\textwidth]{/home/lorenzo/Pictures/Screenshots/Screenshot from 2024-12-30 22-34-23.png}
\end{minipage}
\\  
\\   
\\
\noindent
\begin{minipage}{0.6\textwidth}
%	\vspace{-0.5cm}
Once an IPsec packet with AH Header has been received, the ICV is extracted. Meanwhile the whole packet is normalized, in order to put the packet in the same conditions as it was at the sender in order to correctly compute the Digest. The SPI is also extracted from the AH header to get the correct entry from the SAD, which contains the correct parameters needed to compute the Digest.\\
\\
\textbf{After} having computed the Digest of the normalized packet, it is compared with the extracted
ICV:
\begin{itemize}
    \item If the values are the same \(\rightarrow \) the Sender is authN and the integrity of the packet is
    confirmed.
    \item If the values are \textbf{not} the same \(\rightarrow \) the sender is \textbf{not} authN and/or the packet has been manipulated;
\end{itemize} 
\end{minipage} 
\hspace{0.3cm}
\begin{minipage}{0.4\textwidth}
    \centering
    \includegraphics[width=\textwidth]{/home/lorenzo/Notes/Information System Security/images/Screenshot from 2024-12-30 22-43-13.png}
\end{minipage}
\\    
\\
\\
\noindent
\textbf{Sender authN} is guaranteed by the fact that the SA in the SAD has been implicitly arranged with the sender node. Thus, real authN comes into play when we create the SA.

\subsection{ESP (Encapsulating Security Payload)}
If AH’s security properties are \textbf{not} enough and we want \textbf{confidentiality} \(\rightarrow \) \textbf{ESP} is needed.\\
As AH, ESP comes in two versions:
\begin{itemize}
    \item \textbf{v1 Version}: it achieves \textbf{only confidentiality} using \textbf{DES-CBC}.
    \item \textbf{v2 Version}: it provides also \textbf{payload authN}. The \textbf{advantage} is that the packet dimension is \textbf{reduced} and \textbf{only} one SA is saved.
\end{itemize}
\textcolor{red}{\textbf{N.B.}}  IPsec’s architecture allows us to use AH and ESP at the same time, if needed.
\\
\\
ESP can be used in two modes:
\begin{itemize}
    \begin{minipage}{0.6\textwidth}
    \vspace{0.2cm}
    \item \textbf{Transport Mode}: The \textbf{ESP Header} is inserted in \textbf{between} the IPv4 Header and the Payload, while the \textbf{ESP Trailer} is \textbf{appended} at the end of the whole packet. Finally, \textbf{everything}, from the ESP header up to the trailer, gets then \textbf{encrypted}. 
    \begin{itemize}
        \item \textcolor{green}{\textbf{Pro}}: the payload is hidden.
        \item \textcolor{red}{\textbf{Con}}: the IP Header remains in clear. 
    \end{itemize}
    \end{minipage} 
    \hspace{0.0cm}
    \begin{minipage}{0.4\textwidth}
        \centering
        \includegraphics[width=0.8\textwidth]{/home/lorenzo/Notes/Information System Security/images/Screenshot from 2024-12-30 23-01-28.png}
    \end{minipage}

    \begin{minipage}{0.6\textwidth}
    %	\vspace{-0.5cm}
    \item \textbf{Tunnel Mode}: The first Tunnel is created, then the \textbf{ESP Header} is inserted in between the \textbf{IPv4 Tunnel Header} and the \textbf{IPv4 End-to-End Header}, while the \textbf{ESP Trailer} is appended at the end of the whole packet. As in Transport Mode, \textbf{everything}, from the ESP header up to the
    trailer, gets then \textbf{encrypted}.
    \begin{itemize}
        \item \textcolor{green}{\textbf{Pro}}: the original  Header and the Payload are \textbf{hidden}.
        \item \textcolor{red}{\textbf{Con}}: \textbf{larger}packet size.
    \end{itemize}
    \end{minipage} 
    \hspace{0.0cm}
    \begin{minipage}{0.4\textwidth}
        \centering
        \includegraphics[width=0.8\textwidth]{/home/lorenzo/Notes/Information System Security/images/Screenshot from 2024-12-30 23-09-37.png}
    \end{minipage}
\end{itemize}

\subsection{IPsec Implementation Details}
Many algorithms can be used to implement IPsec, thus two main Crypto-Suites were defined to implement it:
\begin{itemize}
    \item \textbf{VPN-A}: uses \textbf{ESP} with \textbf{3DES-CBC} and \textbf{HMAC-SHA1-96}, this is a kind of \textbf{legacy
    VPN} for compatibility with old systems \(\rightarrow \) less secure.
    \item \textbf{VPN-B}: uses \textbf{ESP} with \textbf{ES-128-CBC} and \textbf{AES-XCBC-MAC-96}, which is the one used in modern systems \(\rightarrow \) more secure.
\end{itemize}
ESP adds also the possibility to use \textbf{NULL Algorithms}, where it is possible to specify \textbf{NULL}
for one of the two parts (authN or confidentiality), but \textbf{not simultaneously} \(\rightarrow \) "protection
against performance" trade-off.

\begin{center}
    \begin{quotebox-grey}{IPsec Partial Protection Against Replay Attacks}
        ESP’s and AH’s \textbf{Sequence Number} provide \textbf{partial} protection from replay attacks since it
        means that IPsec works with a minimum window of 32 packets (64 is the suggested one).\\    \\
        When the SA is created, the sender initializes the Sequence Number to 0. Every time a packet
        is sent, the Sequence Number is incremented by one, and once the value \(2^{31}-1\) is reached, a new SA should be negotiated. The window moves as valid packets are received \(\rightarrow \) \textbf{outside} the window, there is \textbf{no} protection against replay attacks.
      
            \centering
            \includegraphics[width=0.5\textwidth]{/home/lorenzo/Notes/Information System Security/images/image copy 24.png}
    \end{quotebox-grey}   
\end{center}

\subsection{IPsec v3}
\textbf{IPsec v3} makes \textbf{ESP mandatory} and \textbf{AH optional}. It also adds support for single source
multicast. Moreover, to avoid overflow in highly trafficked channels, the \textbf{ESN} (\textbf{Extended Sequence Number}) has been introduced. It uses 8 bytes even though packets inside it are
still 4 bytes \textbf{only}, because \textbf{only} the 4 LSBytes transmitted (default with \textbf{IKEv2}). Finally, v3 adds support for AEAD and clarifies things about SA and SPI to manage faster lookup. The algorithms used in IPsec v3 are the following:
\begin{itemize}
    \item For \textbf{integrity} and \textbf{authN}:
    \begin{itemize}
        \item We may use \textbf{HMAC-MD5-96}.
        \item We must use \textbf{HMAC-SHA1-96}.
        \item We should use \textbf{AES-XCBC-MAC-96}.
        \item We \textbf{must} use \textbf{NULL}, but only if \textbf{ESP} is used.
        \item For a longer digest, we may use:
        \begin{itemize}
            \item \textbf{HMAC-SHA256-128}
            \item \textbf{HMAC-SHA384-192}
            \item \textbf{HMAC-SHA512-256}
        \end{itemize}
    \end{itemize}
    \item For \textbf{privacy}:
    \begin{itemize}
        \item We must use NULL;
        \item We must not use 3DES-CBC;
        \item We should use AES-128-CBC;
        \item We may use AES-CTR;
        \item We should not use DES-CBC;
    \end{itemize}
    \item For \textbf{authN encryption} in \textbf{AEAD Mode} (see below), we can use:
    \begin{itemize}
        \item \textbf{HMAC-SHA256-128}
        \item \textbf{HMAC-SHA384-192}
        \item \textbf{HMAC-SHA512-256}
    \end{itemize}
\end{itemize}
\begin{quotebox}[colframe=blue!10!white, colback=blue!5!white]{TFC (Traffic Flow Confidentiality)}
    \textbf{TFC} is a \textbf{padding technique} used in ESP that puts padding \textbf{after} the payload and before
    the normal padding: this is needed in order to \textbf{not} disclose what is the \textbf{real size} of the payload in the packet.\\     \\    
    Thanks to the length fields (e.g. in TCP, UDP, etc...) the receiver is able to compute the \textbf{original} size of the payload.\\     \\  
    \textbf{Dummy Packets} are a form of Nested Pseudo-Protocol that is introduced by IPsec v3 to keep
    transmitting even in \textbf{absence} of real data to send, which is essential to keep \textbf{confidentiality}.
    Even the Dummy Packets \textbf{must} be encrypted, in order to make them \textbf{indistinguishable} from
    real traffic.
\end{quotebox}

\subsection{Ways to use IPsec}
IPsec can be used to implement various security models:
\begin{itemize}
    \begin{minipage}{0.6\textwidth}
    %	\vspace{-0.5cm}
    \item \underline{\textbf{End-to-End Security}}: IPsec is activated on the End Nodes that are communicating. These nodes need a \textbf{Secure
    Virtual Channel}, so they negotiate a Transport Mode SA. The \textbf{advantage} is that security
    is implemented independently from the rest of the network, which means:
    \begin{itemize}
        \item There is \textbf{no need} to worry about the security of the LAN.
        \item There is \textbf{no need} to worry about the trustworthiness of the gateway’s managing party.
        \item There is \textbf{no need} to worry about the trustworthiness of the WAN.
        \item The \textbf{only} possible attack is a DoS attack.
    \end{itemize}
   
    \end{minipage} 
    \hspace{0.2cm}
    \begin{minipage}{0.4\textwidth}
        \centering
        \includegraphics[width=\textwidth]{/home/lorenzo/Notes/Information System Security/images/Screenshot from 2025-01-01 15-53-53.png}
    \end{minipage}
    \\    
    \\
    \\
    \textcolor{red}{\textbf{Disadvantages}}: 
    \begin{itemize}
        \item IPsec \textbf{must} be installed on both communicating machines, and on some devices the support module may \textbf{not} be natively present (e.g. mobile devices, both
        Android and iOS, embedded systems, etc...).
        \item IPsec is \textbf{slow} and, especially in big LANs, this must be taken into account, as
        there may be the need to have a specif system that efficiently manages the IPsec configurations
        of the various nodes.
    \end{itemize}
    Finally, if the Secure Virtual Channel uses \textbf{ESP} with a Non-NULL Encryption Algorithm \(\rightarrow \) traffic cannot be Sniffed, not even from inside the LAN.

    \begin{minipage}{0.6\textwidth}
    %	\vspace{-0.5cm}
    \item \underline{\textbf{Basic VPN}}: IPsec is activated \textbf{directly} on the Gateways that are protecting the LANs from the WAN. In
    this case, the Gateways create a Tunnel Mode SA between them. The main assumption is
    that the LANs are \textbf{secure} and \textbf{trusted} \(\rightarrow \) attacks are possible \textbf{only} from the WAN.
    \\     \\    
    Yet, this means leaving an open door for \textbf{internal} attacks and it is \textbf{not} possible to perform
\textbf{peer authN} of the real communication endpoint: \textbf{only} the Gateways are authN, which is why
this model is also called \textbf{Site-on-Site VPN}. 
    \end{minipage} 
    \hspace{0.2cm}
    \begin{minipage}{0.4\textwidth}
        \centering
        \includegraphics[width=\textwidth]{/home/lorenzo/Notes/Information System Security/images/Screenshot from 2025-01-01 16-00-17.png}
    \end{minipage}
\begin{quotebox-red}{Beware}
    Since the Gateways \textbf{must} perform the Tunneling for \textbf{all} communication, they could
    get \textbf{overloaded}. Thus, typically, in this case gateways are equipped with some powerful CPU
    or Hardware Accelerators. On the other hand, management is greatly \textbf{simplified}, since \textbf{only}
    the Gateways have to be managed. Moreover, since there is \textbf{no} End-to-End Security \(\rightarrow \) the
    LAN’s traffic can be inspected.
\end{quotebox-red}
\newpage
\begin{minipage}{0.6\textwidth}
%	\vspace{-0.5cm}
\item \underline{\textbf{End-to-End Security with Basic VPN}}: This model is an example of the \textbf{Security in Depth} principle, as it consists of more than one
defense line. Here IPsec is activated two times:
\begin{itemize}
    \item Between the End Nodes (Transport Mode SA) → used for \textbf{peer authN} and \textbf{integrity}.
    \item  Between the Gateways (Tunnel Mode SA) → used to gain \textbf{confidentiality} with encryption;
\end{itemize}
This configuration protects from Sniffing attacks in the WAN, while maintaining the possibility
to inspect the traffic on the LAN.
\end{minipage} 
\hspace{0.2cm}
\begin{minipage}{0.4\textwidth}
    \centering
    \includegraphics[width=0.9\textwidth]{/home/lorenzo/Notes/Information System Security/images/Screenshot from 2025-01-01 16-08-55.png}
\end{minipage}
\\    \\    \\
\textcolor{red}{\textbf{Disadvantages}}: The disadvantages lie in managing of \textbf{all} the Gateway and \textbf{all} the LAN's node at the \textbf{same time}.
\\   \\
\begin{minipage}{0.6\textwidth}
	\vspace{-0.5cm}
\item \underline{\textbf{Secure Gatewar}}: The Secure Gateway model is intended to \textbf{protect Mobile Devices} who need to connect to a \textbf{trusted} LAN. IPsec is deployed on the Mobile Device of the users and on the LAN’s Gateway. A Secure Virtual Channel is created in Tunnel Mode SA. \textbf{All} the traffic coming from the user which is directed to an internal server is protected, and the Gateway can also perform \textbf{peer authN} and \textbf{authorization}.
\end{minipage} 
\hspace{0.2cm}
\begin{minipage}{0.4\textwidth}
    \centering
    \includegraphics[width=0.9\textwidth]{/home/lorenzo/Notes/Information System Security/images/Screenshot from 2025-01-01 16-28-24.png}
\end{minipage}
\begin{minipage}{0.6\textwidth}
%	\vspace{-0.5cm}
\item \underline{\textbf{Secure Remote Access}}: 
The \textbf{Secure Remote Access} model is another example of the principle of \textbf{Security in Depth}, as it consists of more than one defence line. Here IPsec is activated two times:
\begin{itemize}
    \item Between the Mobile Nodes and the Gateway (Tunnel Mode SA) → used for \textbf{peer authN} and \textbf{authorization}.
    \item Between the Mobile Nodes and the End Nodes (Transport Mode SA) → used for \textbf{End to-End Protection}.
\end{itemize}
\end{minipage} 
\hspace{0.3cm}
\begin{minipage}{0.4\textwidth}
    \centering
    \includegraphics[width=0.9\textwidth]{/home/lorenzo/Notes/Information System Security/images/Screenshot from 2025-01-01 16-33-05.png}
\end{minipage}
\end{itemize} 

\subsection{IPsec Key Management}
\textbf{Key Management} is an important component of IPsec since it provides to all parties the
Symmetric Keys used for \textbf{data authN} and (eventually) \textbf{confidentiality}. To distribute these
keys, we can use OOB (passing the keys \textbf{manually}), or, if the number of nodes does not allow
for OOB, we can use some \textbf{Automatic In-Band Protocol} for Key Distribution such as:
\begin{itemize}
    \item \textbf{ISAKMP} (\textbf{Internet Security Association \& Key Management Protocol}):
    it contains the procedures needed to negotiate, set-up, modify and delete a SA.
    The Key Exchange method is \textbf{not fixed} \(\rightarrow \) it can be OOB or an In-Band method that
    uses \textbf{OAKLEY} (\(\rightarrow \) a protocol for authN exchange of symmetric keys).
    \item \textbf{IKE} (\textbf{Internet Key Exchange}): it's a combination of ISAKMP and OAKLEY. It is one of the most \textbf{complex} security
    protocols, because it works in the following way:
    \\
    \begin{minipage}{0.6\textwidth}
    \vspace{0.2cm}
    \begin{enumerate}
        \item The SA is created to protect the ISAKMP exchange. The \textbf{Initiator} takes the initiative to open the IPsec channel towards
        another machine named \textbf{Responder}. In this phase, a \textbf{bidirectional} ISAKMP SA
        is negotiated to protect traffic in both directions, this can be done in \textbf{Main Mode}
        or \textbf{Aggressive Mode}.
        \item The created SA is used to protect the negotiation of the SA needed by IPsec traffic. One of the two nodes acts as an Initiator to create an IPsec SA. This
        can be negotiated in \textbf{Quick Mode} since all the traffic is already protected by the
        ISAKMP SA.
        
    \end{enumerate} 
    \end{minipage} 
    \hspace{0.2cm}
    \begin{minipage}{0.4\textwidth}
       
        \includegraphics[width=0.7\textwidth]{/home/lorenzo/Pictures/Screenshots/Screenshot from 2025-01-01 16-44-13.png}
    \end{minipage}
   \begin{quotebox-yellow}{Modes of Operation}
   \begin{itemize}
    \item \textbf{Main Mode}: 6 messages → quite \textbf{slow}. It can protect the parties’ identities.
    \item \textbf{Aggressive Mode}: 3 messages, it is faster than Main Mode. But it cannot protect the parties’ identities.
    \item \textbf{Quick Mode}: 3 messages, it is used \textbf{only} to negotiate the IPsec SA.
    \item \textbf{New Group Mode}: 2 messages, it is used to \textbf{communicate} to other peer to inform it about a \textbf{change} in the algorithm or the key that is being used to protect the traffic.
   \end{itemize}
   \end{quotebox-yellow}
   \begin{quotebox-grey}{AuthN Method}
     When a SA is created, we \textbf{must} specify which \textbf{AuthN Method} is needed:
     \begin{itemize}
        \item \textbf{Digital Signature}: non-repudiation of the IKE negotiation \(\rightarrow \) it is \textbf{not} possible to deny the request to open the Secure Channel.
        \item \textbf{Public Key Encryption}: identity protection provided by the Aggressive Mode.
        \item \textbf{Revised Public Key Encryption}: faster since only two PK operations are needed.
        \item \textbf{Pre-Shared Key}: the party’s ID may \textbf{only} be its own IP address \(\rightarrow \) may create problems for Mobile Users.
     \end{itemize}
   \end{quotebox-grey}
\end{itemize}

\subsection{VPN Concentrator}
Since, IPsec, nowadays, is used to create mainly Site-to-Site VPN, there is the problem of the \textbf{performance} of Gateways. For that reason, \textbf{VPN Concentrator} were created. They are a
special purpose hardware appliance that acts as a \textbf{terminator} of IPsec Tunnel.\\   \\
VPN Concentrators are used for remote access of single users or to create Site-to-Site VPN.
Since they are implemented in hardware, VPN Concentrators have a \textbf{very high performance}
with respect to the \textbf{low cost}.

\subsection{IPsec Conclusions}
IPsec can \textbf{only} be applied to \textbf{Unicast packets}, since it needs to identify the peer and exchange
a key. Moreover, since reciprocal authN is needed, IPsec is applied \textbf{only} between parties that
activated a SA by means of Shared Keys or by means of X.509 certificates.\\
IPsec is good \textbf{only} for "closed" groups \(\rightarrow \) IPsec \textbf{cannot} be used to create services that do
\textbf{not} need to client authN (e.g. e-commerce).
\\
\noindent{\color{gray!50}\rule{\textwidth}{0.5pt}}
\section{Service Protocols (In)security}
IP addresses are \textbf{not} authN and packets are \textbf{not} protected (lack of \textbf{integrity}, \textbf{authN}, \textbf{confidentiality} and protection against Replay attacks) when IPsec is \textbf{not} used. Therefore, \textbf{all} the protocols that use IP as a carrier can be attacked.

\subsection{ICMP Security}
\textbf{ICMP} (\textbf{Internet Control \& Management Protocol}) is a vital protocol for network management and, for this reason, many attacks are possible because it does \textbf{not} provide authN.
This means that the following ICMP functions can be exploited to perform attacks:
\begin{itemize}
    \item \textbf{ICMP Echo Request/Reply}: can be used to perform ping flooding or ping bombing → both DoS attacks.
    \item \textbf{Destination Unreachable}: since packets are \textbf{not} authN, DoS attacks can be performed:
    fake nodes could send Destination Unreachable to the sender, which in turn will close the communication thinking that the destination is truly unreachable.
    \item \textbf{Source Quench} (deprecated from 2012): it was a mechanism meant to provide congestion control. DoS attacks could be performed in the following way: since senders \textbf{must}
    react to ICMP Source Quench messages, intermediate nodes’ buffers will get filled up by
    the responses, \textbf{slowing} transmission on the connection. This was possible mainly because
    ICMP Source Quench messages were \textbf{not} authN.
    \item \textbf{Redirect}: these ICMP messages can be sent by any intermediate node when they detect
    that a packet has taken the wrong path. This make \textbf{Logical MITM attacks} possible: an attacker could redirect the sender to a malicious node under its control and there it can perform the MITM attack.
    \item \textbf{Time Exceeded for a Datagram}: it is an ICMP message normally sent by intermediate
    nodes when they process a packet with \textbf{TTL = 0}. Receiving this message usually means
    that there are loops in the routing plan \(\rightarrow \) the source node of the packet closes the
    connection \(\rightarrow \) attackers could fake this message causing a DoS for the sender.
    \begin{quotebox-yellow}{}
        These attacks are \textbf{always} possible, since it is \textbf{not} possible to protect ICMP with IPsec.
    \end{quotebox-yellow}
\end{itemize}
\begin{quotebox-grey}{Smurfing Attack}
    \begin{minipage}{0.6\textwidth}
    %	\vspace{-0.5cm}
    A \textbf{Smurfing attack} can be performed with ICMP. Smurfing is a kind of DoS attack, where a
    target is attacked by creating an ICMP Echo Request (a.k.a. ping) with \(IP_A\) as the Source
    Address, while the destination is the Broadcast Address of the whole network. The nodes of
    the network, upon receiving the ICMP Echo Request, will start sending ICMP Echo Replies
    to the source of the request → the victim \textbf{A}, which will get overloaded with messages. At the
    same time, the \textbf{Reflector Network} will be quite busy dealing with the huge amount of ICMP
    traffic \(\rightarrow \) both the victim and the Reflector Network suffer the DoS attack.
    
    \end{minipage} 
    \hspace{0.3cm}
    \begin{minipage}{0.4\textwidth}
        \centering
        \includegraphics[width=0.9\textwidth]{/home/lorenzo/Notes/Information System Security/images/Screenshot from 2025-01-01 17-33-10.png}
    \end{minipage}
    \\  \\
    \\
    To prevent this attack, we have to first understand where the attack is coming from:
    \begin{itemize}
        \item \textbf{Extern Attack}: the network’s Border Routers should \textbf{reject} incoming IP Broadcast packets.
        \item \textbf{Internal Attack}: since broadcast \textbf{cannot} be disabled in LAN (it is required by protocols), the only way to stop a Smurfing attack is to identify the attacker via Network Management Tools and then try to stop its machine.
    \end{itemize}
\end{quotebox-grey}
\begin{quotebox-grey}{Fraggle Attack}
    \begin{minipage}{0.6\textwidth}
    \vspace{-1.5cm}
    \textbf{Fraggle attacks} are very similar to smurfing, since they follow the \textbf{same tactic}. However,
    Fraggle attacks are performed with \textbf{UDP} rather than \textbf{ICMP}. Nowadays, the UDP Echo
    Request are \textbf{disabled} by default, thus, this type of DoS attacks relies on the number of clients
    inside the reflector network that have UDP Echo Request \textbf{enabled}. 
    \end{minipage} 
    \hspace{0.3cm}
    \begin{minipage}{0.4\textwidth}
        \centering
        \includegraphics[width=0.9\textwidth]{/home/lorenzo/Notes/Information System Security/images/Screenshot from 2025-01-01 17-37-02.png}
    \end{minipage}

\end{quotebox-grey}

\subsection{ARP Poisoning}
\textbf{ARP Poisoning} is performed by exploiting the fact that nodes accept ARP Reply \textbf{even if}
they did \textbf{not} send an ARP Request, this is because rejecting it could mean avoiding the need
to send an ARP Request in the future, when contacting that node.\\ 
Thus, it is possible to send \textbf{unsolicited wrong} ARP Replies to the nodes, which will overwrite
their \textbf{static} ARP Table Entries with \textbf{dynamic} ones.\\
ARP Poisoning attacks are possible because \textbf{most} protocols do \textbf{not} check that the Source L2 Address inside the ARP field matches with the one in the Source Field of the Ethernet frame.
\subsection{TCP SYN Flooding}
\begin{minipage}{0.6\textwidth}
%	\vspace{-0.5cm}
Attackers could start a 3-Way Handshake to open a TCP connection, but \textbf{never} send the final
ACK back. This makes the server fill up its table of open TCP connections, meanwhile the
attacker can keep going back to the first step just by sending another SYN packet.\\
At some point the table will be full and the server will \textbf{not} be able to handle any new connections
attempts \(\rightarrow \) DoS for the clients who wish to connect to the server.
\end{minipage} 
\hspace{0.5cm}
\begin{minipage}{0.4\textwidth}
    \centering
    \includegraphics[width=\textwidth]{/home/lorenzo/Notes/Information System Security/images/Screenshot from 2025-01-01 17-44-53.png}
\end{minipage}
\\
\\
To protect against \textbf{SYN Flooding} it is possible to:
\begin{itemize}
    \item \textbf{SYN Interceptor}: a router could bu put in "front" of the server with \textbf{aggressive} but
    \textbf{risky} timeout. The router would \textbf{only} have to wait for the 3-way handshake to complete
    in due time, successful handshakes would then be transferred to the server.
    \vspace{-0.3cm}
    \begin{figure}[H]
        \centering
        \includegraphics[width=0.35\textwidth]{/home/lorenzo/Notes/Information System Security/images/Screenshot from 2025-01-01 17-52-20.png}
        \caption{SYN Interceptor or Firewall Relay}
    \end{figure}
    \item \textbf{SYN Monitor}: a router could be used to monitor pending requests and eventually kill
    them.
    \begin{figure}[H]
        \centering
        \includegraphics[width=0.35\textwidth]{/home/lorenzo/Notes/Information System Security/images/Screenshot from 2025-01-01 17-58-09.png}
        \caption{SYN Monitor or Firewall Gateway}
    \end{figure}
    \item \textbf{Decrease the Timeout}: at the end of the timeout, the server will check its connection
    table and \textbf{discard} pending connections. However, the server risks to delete connections
    requests coming from \textbf{valid} but \textbf{slow} clients.
    \item \textbf{Increase the Table Size} → could easily be avoided by sending more requests.
\end{itemize}
\begin{quotebox}[colframe=blue!10!white, colback=blue!5!white]{SYN Cookie}
    \textbf{SYN Cookies} are the \textbf{only} thing that can \textbf{completely protect} against SYN flooding attack.
    A SYN Cookie uses the TCP Sequence Number of the SYN-ACK packet to transmit a \textbf{cookie}
    to the requesting client. This cookie is later used to \textbf{recognize} a client that has already sent
    the SYN \textbf{without} storing any info about them on the server.
\end{quotebox}

\subsection{DNS Security}
\begin{minipage}{0.6\textwidth}
\vspace{-0.5cm}
The DNS provides translation from \textbf{name} (URL) → \textbf{address} (IP) and vice-versa if needed.
The DNS is an \textbf{essential service} that works by means of \textbf{queries} (over port \textbf{53/UDP}) and
\textbf{zone transfers} (over \textbf{53/TCP}), which allow the transfer of information between DNS Servers.
By itself, DNS does \textbf{not} provide any kind of security.
\end{minipage} 
\hspace{0.5cm}
\begin{minipage}{0.4\textwidth}
    \centering
    \includegraphics[width=0.7\textwidth]{/home/lorenzo/Notes/Information System Security/images/Screenshot from 2025-01-01 18-10-05.png}
\end{minipage}
\begin{center}
    
\begin{quotebox-grey}{DNS Shadow Server}
    \begin{minipage}{0.6\textwidth}
    	\vspace{-0.5cm}
    An attacker could Sniff the request that the victim sent to the Local DNS \(\rightarrow \) fake IP address
    can be provided to the victim. An attacker could also Sniff the request going from the Local
    DNS to the Root DNS \(\rightarrow \) fake responses can be provided to the Local DNS \(\rightarrow \) fake answer
    can be provided to \textbf{all} the clients of the network that query the Local DNS.
    \end{minipage} 
    \hspace{0.3cm}
    \begin{minipage}{0.4\textwidth}
        \centering
        \includegraphics[width=0.9\textwidth]{/home/lorenzo/Notes/Information System Security/images/Screenshot from 2025-01-01 18-14-48.png}
    \end{minipage}
    \\   \\
    Moreover, Microsoft implemented on Windows a cache for DNS, which is \textbf{violating} the general
rules for DNS. In this case, \textbf{DNS Shadow Server} attacks are much \textbf{more effective} because cache on client is \textbf{not} often updated.
\end{quotebox-grey}
\end{center}


\begin{center}
   
    \begin{quotebox-grey}{DNS Cache Poisoning}
        Every DNS Server has a cache to fasten up response times. With \textbf{DNS Cache Poisoning}, an
        attacker could create a domain (e.g. www.attacker.net) and a \textbf{nameserver}. This nameserver
        will answer to anyone who contacts it with the \textbf{real} address of the server \textbf{plus} some other \textbf{wrong} information.
        \\
        If the victim’s nameserver is \textbf{not} properly configured, it will accept the additional answers and \textbf{overwrite} the information already available in its cache.

            \centering
            \includegraphics[width=0.4\textwidth]{/home/lorenzo/Notes/Information System Security/images/Screenshot from 2025-01-01 18-21-48.png}
    \end{quotebox-grey}
\end{center}
\begin{quotebox-grey}{DNS Cache Poisoning v2}
    \begin{minipage}{0.6\textwidth}
    \vspace{-0.5cm}
    In this version, the attacker configures a DNS Client and asks for the victim’s \textbf{Recursive
    Nameserver} for the resolution of the address. But since the search will take time, the attacker
    can send the answer to its own request providing a \textbf{fake Source Address} (as if it was sent
    from the Authoritative Nameserver).\\
    This attack is much \textbf{more effective} since it is \textbf{not} necessary for the victim to be tricked into requesting the resolution of the attacker’s nameserver. 
    \end{minipage} 
    \hspace{0.3cm}
    \begin{minipage}{0.4\textwidth}
        
        \includegraphics[width=0.9\textwidth]{/home/lorenzo/Pictures/Screenshots/Screenshot from 2025-01-01 18-26-14.png}
    \end{minipage}
\end{quotebox-grey}
\begin{quotebox-grey}{DNS Flash Crowd}
    \begin{minipage}{0.6\textwidth}
    \vspace{-2cm}
    The \textbf{DNS Flash Crowd} is the \textbf{only} possible way to perform a DoS attack on a DNS Server.
    The DNS Server is targeted by a huge number of clients and servers that block access to \textbf{all}
    the domains related to that DNS Server, making it \textbf{unreachable}. 
    \end{minipage} 
    \hspace{0.3cm}
    \begin{minipage}{0.4\textwidth}
        \centering
        \includegraphics[width=\textwidth]{/home/lorenzo/Notes/Information System Security/images/Screenshot from 2025-01-01 18-28-17.png}
    \end{minipage}
\end{quotebox-grey}

\subsection{NAT Security}
Name-Address Translation System is used by numerous application and it's typically implemented in DNS (Domain Name System)
architectures.\\
\\
\textbf{The process:}
\begin{enumerate}
    \item \textbf{App \#1 Initiates a Request}: Wants to resolve a domain name (e.g., www.example.com)
    to an IP address. It sends the request to the OS through the Name Switch.
    \item \textbf{Name Switch}: The Name Switch in the OS determines how the domain name should be resolved. It checks in the following in order:
    \begin{itemize}
        \item Local Data (e.g., /etc/hosts): If the name exists in local configuration files, it uses
        this data.
        \item If the name is not found locally, it forwards the query to the Resolver.
    \end{itemize}
    \item \textbf{Resolver}: The Resolver handles DNS queries on behalf of the OS \(\rightarrow \) It forwards the query to a (Recursive) Caching Server for further resolution.
    \item \textbf{Recursive Caching Server}: checks its Cache to see if it already has the requested domain name’s IP address:
    \begin{itemize}
        \item Cache Hit: If the result exists in the cache, it returns the IP address to the resolver.
        \item Cache Miss: If the result is not in the cache, it queries the authoritative DNS servers.
    \end{itemize}
    \item \textbf{Zone Master (Primary) and Zone Slave (Secondary)}:
    \begin{itemize}
        \item The Zone Master responds with the authoritative answer.
        \item The Zone Slave (Secondary) might also serve the query if it holds a copy of the zone
        file.
    \end{itemize}
    \item \textbf{Response Propagation}: Once the Recursive Caching Server obtains the correct IP
    address, it sends the result back to the Resolver. The Resolver delivers the resolved IP address to the Name Switch, which forwards it to App \#1.
\end{enumerate}

\noindent
The DNS Infrastructure can be attacked in the following ways:
\\
\begin{minipage}{0.6\textwidth}
	\vspace{0.3cm}
\begin{itemize}
    \item \textbf{Local Data Corruption}: an attacker could get access to a client and try to insert
    among its local data some information to produce a \textbf{wrong} name-address translation.
    \item \textbf{Unauthorized Configuration}: Name Switch or the Resolver could be reconfigured to point to a \textbf{fake} server that provides \textbf{wrong} answers.
    \item \textbf{Server Impersonation}: since DNS does \textbf{not} have any kind of authN, communication
    between the Resolver and the \textbf{Caching Server} or the \textbf{Zone Server} are vulnerable to
    MITM attacks.
    \item \textbf{Cache Poisoning}: every time there is a cache, it can be poisoned.
    \item \textbf{Dynamic Update Exploit}: unauthorized nodes could perform updates.
\end{itemize} 
\end{minipage} 
\hspace{0.3cm}
\begin{minipage}{0.4\textwidth}
    \centering
    \includegraphics[width=\textwidth]{/home/lorenzo/Notes/Information System Security/images/Screenshot from 2025-01-01 19-06-26.png}
\end{minipage}

\noindent
Apart from attacks against the nameservers, DNS has got a \textbf{user privacy} problem for the
queries as they can be: read while in transit or read and logged by the nameserver.\\
Therefore, to improve DNS and \textbf{protect the queries}, the following protocols were created:
\begin{itemize}
    \item \textbf{DoT (DNS-over-TLS)}: query and response are \textbf{encapsulated} in a Secure TLS Tunnel,
    yet it is still evident that the ongoing communication is a DNS exchange, thus it can be
    attacked.
    \item \textbf{RFC-8484 (DNS-over-HTTPS)}: query and response are part of a normal HTTPS
    exchange → externally it looks like visiting a secure web page.
\end{itemize}



\subsection{Protection from IP Spoofing}
It is \textbf{not} possible to prevent a host from changing its own address, but it is possible to try and limit the diffusion of \textbf{IP Spoofing}.\\
To protect the internal network from external impostors and protect the external network form internal impostors, the following RFCs suggest how to \textbf{increase} the protection from IP
spoofing:
\begin{itemize}
    \item \textbf{RFC-2827}: "Network ingress filtering: defeating DoS attacks which employ IP source
    address spoofing".
    \item \textbf{RFC-3704}: "Ingress filtering multihomed networks".
    \item \textbf{RFC-3013}: "Recommended ISP security services and procedures".
\end{itemize}
\begin{minipage}{0.5\textwidth}    
        \centering
        \includegraphics[width=0.8\textwidth]{/home/lorenzo/Notes/Information System Security/images/Screenshot from 2025-01-01 18-48-26.png}       
\end{minipage} 
\hspace{0.5cm}
\begin{minipage}{0.5\textwidth}   
        \centering
        \includegraphics[width=0.8\textwidth]{/home/lorenzo/Notes/Information System Security/images/Screenshot from 2025-01-01 18-49-02.png}
\end{minipage}
\\   \\    \\
By placing filters on the \textcolor{red}{\textbf{red}} and \textcolor{blue}{\textbf{blue}} lines, it is possible to protect against both external and internal impostors. If \textbf{all} providers would apply this simple rule, it would be possible to avoid IP spoofing \textbf{globally}.