\chapter{Security Implementations in OSI Levels}
\begin{quotebox-yellow}{}
    In the ISO/OSI stack, there is \textbf{not} a single optimal level in which we can implement security. Typically, L6 (Presentation) is the \textbf{only} level in which security measures are \textbf{not} useful.
\begin{itemize}
    \item The \textbf{higher} we go in the stack, the more our security features are \textbf{effective}, but we \textbf{slow}
    down our system and we leave more room for DoS attacks.
    \item The \textbf{lower} we go in the stack, the more \textbf{quickly} we detect attacks, but the \textbf{fewer} are he data available to us to detect them.
\end{itemize}
\end{quotebox-yellow}
\section{DHCP Security}
When L3 is reached, one of the first protocols that is activated is the \textbf{DHCP (Dynamic Host Configuration Protocol)}. Unfortunately, DHCP is \textbf{not} capable of performing peer authN
and uses \textbf{broadcast} packets that carry: IP address; netmask; default gateway; local DNS and local DNS suffix (\(\rightarrow \) \textbf{sensitive data}).\\   
\\    
Since DHCP Requests are L2 broadcast frames, an attacker could activate a \textbf{Fake DHCP
Server} by staying in the same broadcast domain of the victim and Sniffing the victim’s DHCP Request.\\    
\\    
Possible attacks that can be performed by a Fake DHCP Server are:
\begin{itemize}
    \item \textbf{DoS}: done by providing a wrong network configuration to the victim, denying access to the network.
    \item \textbf{MIMT}:  a valid IP address is provided to the victim, but it will be a part of a /30 subnet \(\rightarrow \) \textbf{only} two IP addresses can be assigned:
\end{itemize}